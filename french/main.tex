%%%%%%%%%%%%%%%%%%%%%%%%%%%%%%%%%%%%%%%%%
% "ModernCV" CV and Cover Letter
% LaTeX Template
% Version 1.1 (9/12/12)
% Important note:
% This template requires the moderncv.cls and .sty files to be in the same 
% directory as this .tex file. These files provide the resume style and themes 
% used for structuring the document.
%
%%%%%%%%%%%%%%%%%%%%%%%%%%%%%%%%%%%%%%%%%


%----------------------------------------------------------------------------------------
%	PACKAGES AND OTHER DOCUMENT CONFIGURATIONS
%----------------------------------------------------------------------------------------

\documentclass[12pt,a4paper,sans]{moderncv} % Font sizes: 10, 11, or 12; paper sizes: a4paper, letterpaper, a5paper, legalpaper, executivepaper or landscape; font families: sans or roman

\moderncvstyle{classic} % CV theme - options include: 'casual' (default), 'classic', 'oldstyle' and 'banking'
\moderncvcolor{purple} % CV color - options include: 'blue' (default), 'orange', 'green', 'red', 'purple', 'grey' and 'black'

\usepackage{lipsum} % Used for inserting dummy 'Lorem ipsum' text into the template
\usepackage[utf8]{inputenc}  
\usepackage[T1]{fontenc}
\usepackage{color}
\usepackage{lastpage}

\rfoot{\textit{\small{\thepage/\pageref{LastPage}}}}
\usepackage[scale=0.90]{geometry} % Reduce document margins
%\setlength{\hintscolumnwidth}{3cm} % Uncomment to change the width of the dates column
%\setlength{\makecvtitlenamewidth}{10cm} % For the 'classic' style, uncomment to adjust the width of the space allocated to your name

%----------------------------------------------------------------------------------------
%	NAME AND CONTACT INFORMATION SECTION
%----------------------------------------------------------------------------------------

\firstname{Walid} % Your first name
\familyname{Saad} % Your last name

% All information in this block is optional, comment out any lines you don't need
\title{Ingénieur en Informatique\newline
}
\address{}{Ariana, Tunisie}
\mobile{(+216) 22227151}
\email{walid.sa3d@gmail.com}
\social[linkedin]{linkedin.com/in/walidsa3d}
%\social[github]{github.com/walidsa3d}
%\social[twitter]{walidsa3d}
\photo[85pt][0.0pt]{walidpic.jpg} 
%\quote{"Viam aut inveniam aut faciam"}
 % The first argument is the url for the clickable link, the second argument is the url displayed in the template - this allows special characters to be displayed such as the tilde in this example

%----------------------------------------------------------------------------------------

\begin{document}

\makecvtitle % Print the CV title

%----------------------------------------------------------------------------------------
%	EDUCATION SECTION
%----------------------------------------------------------------------------------------

\section{Formation}

\cventry{}{Diplôme National D'Ingénieur En Informatique}{ESPRIT}{Tunis}{\textit{}}{-Obtenu en septembre 2014}% Arguments not required can be left empty
\cventry{}{Classes Préparatoires Aux Etudes D'Ingénieur}{IPEIM}{Monastir}{\textit{}}{
-Option Physique-Chimie.
}

%----------------------------------------------------------------------------------------
%	COMPUTER SKILLS SECTION
%----------------------------------------------------------------------------------------

\section{Compétences Techniques}
\cvitem{Programmation}{Java, Python, Unix Shell, SQL, XML}
\cvitem{Conception}{UML}
\cvitem{Web}{HTML/ CSS, RDF/OWL/SPARQL, PHP, Javascript, JQuery, Twitter Bootstrap, AngularJS, Wordpress}
\cvitem{Mobile}{J2ME, Android}
\cvitem{SGBD}{Oracle, MySQL, PostgreSQL, SQLite}
\cvitem{OS}{Windows, Ubuntu, CentOS, ArchLinux}
\cvitem{Outils}{Eclipse, CheckStyle, NetBeans, StarUML, Git, Maven}
%----------------------------------------------------------------------------------------
%	LANGUAGES SECTION
%----------------------------------------------------------------------------------------

\section{Langues}
\cvitemwithcomment{Arabe}{Langue maternelle}{}
\cvitemwithcomment{Francais}{Bilingue}{}
\cvitemwithcomment{Anglais}{Très Bon Niveau}{}
\cvitemwithcomment{Allemand}{Notions De Base}{}

%----------------------------------------------------------------------------------------
%	EXPERIENCE SECTION
%----------------------------------------------------------------------------------------

\section{Stages}
\cventry{Février 2014}{Stage PFE}{\textsc{EspriTech}}{Tunis}{}{Développement d'une application pour la gestion des dispositifs dans une maison intelligente.
\begin{itemize}
\item Détection automatique et classification des dispositifs en temps réel.
\item Interface web qui permet l'affichage et la recherche des dispositifs selon plusieurs critères.
\end{itemize}
\textcolor{blue}{\textsc{Mots-Clés}}: JBoss, JEE, JSF+Primefaces, XML, Atmosphere Framework, Apache Jena, Maven, UPnP.\newline
%\textcolor{blue}{\textsc{Code-Source}}: github.com/walidsa3d/smart-annotator
}
\cventry{Aout 2012}{Stage D'Immersion en Entreprise}{\textsc{Groupe Chimique Tunisien}}{Gafsa}{}{Développement d'un site web pour l'entreprise.\newline
\textcolor{blue}{\textsc{Mots-Clés}}: WordPress, PHP.
}
\section{Projets}
\cventry{Octobre 2013}{Projet Spring}{\textsc{}}{Tunis}{}{Développement d'une application web de gestion des contacts.\newline
\textcolor{blue}{\textsc{Mots-Clés}}: Apache Tomcat, Spring MVC, Spring Data, Spring Security, JPA, Web Services, AngularJS, Twitter Bootstrap, MySQL}
\cventry{Novembre 2013}{Projet Administration et Sécurité des réseaux}{\textsc{ESPRIT}}{Tunis}{}{Installation, configuration et sécurisation d'un réseau d'entreprise.
\begin{itemize}
\item Installation et configuration des serveurs web (ssl et virtualhosts), ftp et mail.
\item Supervision des serveurs avec l'outil Nagios
\item Sécurisation de l'accès avec VPN et SSH
\item Haute disponibilité des serveurs avec HeartBeat
\end{itemize}
\textcolor{blue}{\textsc{Mots-Clés}}: Ubuntu Server, Windows Server, Apache Sever, vsfptd, Postfix, Dovecot, DNS, DHCP, openSSH, openVPN, VMware Workstation.
}
\cventry{Février 2013}{Projet Java/JEE}{\textsc{Esprit}}{Tunis}{}{Développement  d'une application de gestion d'une agence de voyage.
\begin{itemize}
  \item Travail d'équipe selon la méthodologie Scrum
  \item Gestion des voyageurs, hotels, vols et offres
  \item Création d'un système d'audit pour assurer le tracabilité des opérations.
  \item Création d'un système d'authentification/autorisation basé sur les rôles/privilèges.
\end{itemize} 
\textcolor{blue}{\textsc{Mots-Clés}}: JBoss, JEE, EJB, JPA, Hibernate, JSF, Primefaces, JUnit, Swing, JavaMail, MySQL.\newline
%\textcolor{blue}{\textsc{Code-Source}}: github.com/walidsa3d/hajj-travel-agency
}
\cventry{Octobre 2012}{Mini-Projet Java}{\textsc{Esprit}}{Tunis}{}{Développement d'un logiciel éducatif : Quiz Generator
\begin{itemize}
\item Génération de tests aléatoires à partir d'une base de données.
\item Evaluation des tests et envoi des résultats par mail.
\item Génération des rapports des tests à l'aide de la bibilothèque JFreeChart.
\end{itemize}
\textcolor{blue}{\textsc{Mots-Clés}}: Java, JDBC, Swing, JFreeChart, Apache Commons, MySQL.}
\cventry{Février 2012}{Mini-Projet C}{\textsc{Esprit}}{Tunis}{}{Développement d’un jeu sur console : TicTacToe.}
\section{Contributions}
\cventry{Avril 2014}{Portail ESPRIT JUG}{\textsc{}}{Tunis}{}{Développement du portail web de ESPRIT Java User Group. \newline
\textcolor{blue}{\textsc{Mots-Clés}} : JBoss, JEE, JSF+Primefaces, EJB, JPA+Hibernate, REST Web Services, MySQL.
}

%----------------------------------------------------------------------------------------
%	INTERESTS SECTION
%----------------------------------------------------------------------------------------

\section{Activités et Centres D'Intérêt}

%\renewcommand{\listitemsymbol}{-~} % Changes the symbol used for lists
\cvlistitem{Ancien membre du club ESPRIT Reporters, éditeur de la magazine de ESPRIT.}
\cvlistitem{Ancien membre du club ESPRIT Libre, le club des logiciels libres.}
%-------------------------------------------------------------------------------
\end{document}Z
%\homepage{github.com/walidsa3d}{github.com/walidsa3d} % The first argument is the url for the clickable link, the second argument is the url displayed in the template - this allows special characters to be displayed such as the tilde in this example
%----------------------------------------------------------------------------------------

